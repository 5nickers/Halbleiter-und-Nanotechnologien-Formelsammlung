\input{header.tex}


\begin{document}

\maketitle

Dieser Text ist unter dieser \href{http://creativecommons.org/licenses/by-nc-sa/4.0/}{Creative Commons} Lizenz veröffentlicht.

\textcolor{red}{Ich erhebe keinen Anspruch auf Vollständigkeit oder Richtigkeit. Falls ihr Fehler findet oder etwas fehlt, dann meldet euch bitte über den Emailkontakt.}

\tableofcontents


\newpage


Ich habe keine Formeln aus vorigen Übungen inkludiert, da diese recht gut im Skript zusammengefasst sind.



\section{Aus Übung 5}

\begin{align*}
\intertext{Zustandsdichten im k-Raum:} 
D(k)\d k &= \frac{\pi k^2}{\pi^3} \d k 
\intertext{Zustandsdichte im Energieraum:}
D(E) \d E &= \frac{4 \pi \cdot \left( 2m \right)^{3/2)} \cdot \sqrt{E}}{h^3}
\intertext{Dichte der Zustände:}
n &= \int D(E) \d E
\intertext{Wahrscheinlichkeit eines besetzten Elektronenzustandes}
f_h(E) &= \frac{1}{1 + e^{\frac{E - E_F}{kT}}}
\end{align*}



\section{Aus Übung 6}

\begin{align*}
\intertext{Zustandsdichte}
D(E) &= \frac{\d N}{\d E} = \frac{1}{2 \pi^2} \cdot \left( \frac{2m}{\hbar^2} \right)^{\frac{3}{2}} \cdot \sqrt{E} \cdot V
\intertext{Elektronendichte}
n &= \frac{N}{V}
\intertext{Fermienergie}
E_F &= \left( 3 \pi^2 \right)^{\frac{2}{3}} \cdot \frac{\hbar^2}{2m} \cdot n^{\frac{2}{3}} = \frac{E_C - E_V}{2} + \frac{3kT}{4} \cdot \ln\left( \frac{m_h*}{m_e*} \right) \qquad \text{\textcolor{red}{im Skript mit +}}
\intertext{Fermitemperatur}
T_F &= \frac{E_F}{k}
\intertext{Innere Energie von Fermigas}
U &= \frac{3}{5} \cdot N \cdot E_F
\intertext{Teilchendruck}
P &= \frac{\p U}{\p V} = \frac{2}{5} \cdot n \cdot E_F
\intertext{Zustanddichte der Elektronen}
N_C &= 2 \cdot \left( \frac{kT}{2 \pi \hbar} \right)^{3/2} \cdot m*_e^{3/2}
\intertext{Zustanddichte der Löcher}
N_V &= 2 \cdot \left( \frac{kT}{2 \pi \hbar} \right)^{3/2} \cdot m*_h^{3/2}
\intertext{Teilchenenergie (masseabhängig)}
E &= \frac{\hbar^2 \cdot k^2}{2m^*}
\end{align*}


\section{Aus Übung 7}


\begin{align*}
\intertext{Besetzungsdichte bei einem dotierten Halbleiter}
n_d &= N_d \cdot \frac{1}{1 + \half e^{\frac{E - E_F}{kT}}}
\intertext{Zustandsdichte $N_c$}
N_c &= 2 \cdot \left( \frac{2 \pi \cdot m^* \cdot kT}{h^2} \right)^\frac{3}{2}
\intertext{die Elektronenkonzentration dazu ist dann}
n &= N_c \cdot e^{- \frac{E_c - E_F}{kT}}
\intertext{Raumladungsweite bei einem Schottky-Kontakt}
w &= \sqrt{\frac{2 \cdot \epsilon_r \cdot \epsilon_0 \cdot \left( V_{bi} + V_R \right)}{e \cdot N_d}} \qquad \text{Dabei ist  $V_R$ die Gatespannung mit - am Gate}
\intertext{Die Kapazität}
C' &= \frac{C}{A} = \frac{\epsilon_0 \epsilon_r}{w} = \sqrt{\frac{e \cdot N_d \cdot \epsilon_0 \epsilon_r}{\left( V_{bi} + V_R \right)^2}}
\end{align*}


\section{Aus Übung 8}

\begin{align*}
\intertext{Sättigungsstromdichte:}
j_s &= A \cdot T^2 \cdot e^{-e \cdot \frac{\phi_{SB} - \Delta \phi}{kT}}
\intertext{Widerstand:}
R &= \frac{l \phi}{A} = \frac{l}{n \cdot e \cdot \mu \cdot A}
\intertext{Der y-Achsen Abschnitt bei einer Transmissionslinie ist $2R_C$. Es gilt weiter:}
l_0 &= 2 \cdot \frac{R_s}{r_s} \cdot l_T \approx 2 \cdot l_T
\intertext{effektive Kontaktfläche:}
A_{eff} &= A \cdot l_T
\intertext{spezifischer Kontakwiderstand:}
\rho_e &= R_c \cdot A
\intertext{Schottky Barriere}
\phi_{SB} &= V_{bi} + \phi_n
\end{align*}


\section{Aus Übung 9}

\begin{figure}[h]
	\centering
	\includegraphics[scale=0.1]{U1_1.jpg}
\end{figure}


\begin{align*}
\intertext{Pinch-Off Spannung}
V_P &= \frac{a^2 \cdot e \cdot N_d}{2 \cdot \epsilon_0 \cdot \epsilon_r}
\intertext{Pinch-Off Strom}
I_P &= \frac{z \cdot \mu \cdot q^2 \cdot N_d^2 \cdot a^3}{6 \cdot \epsilon_0 \epsilon_r \cdot L}
\intertext{Sättigungsstrom}
I_{Sä} &= I_P \cdot \left[ \frac{3 \cdot \left( V_P - V_g - V_{bi} \right)}{V_P} - \frac{2 \cdot \left( V_P^{3/2} - \left( V_g + V_{bi} \right)^{3/2} \right)}{V_P^{3/2}} \right]
\intertext{Drain-Spannung}
V_{DS} &= V_P - V_{bi} - V_g
\intertext{Steilheit}
g_{m} &= I_P \cdot \left[ \frac{-3}{V_P} + \frac{3 \cdot \sqrt{V_g + V_{bi}}}{V_P^{3/2}} \right]
\end{align*}



\section{Transistoren}


\subsection*{Kollektorschaltung}

\begin{figure}[h]
	\centering
	\includegraphics[scale=0.9]{Kollektorschaltung.jpg}
\end{figure}


Spannungsverstärkung kleiner 1, aber sehr hohe Stromverstärkung


\newpage

\subsection*{Emitterschaltung}

\begin{figure}[h]
	\centering
	\includegraphics[scale=0.9]{Emitterschaltung.jpg}
\end{figure}


kleine Stromverstärkung, aber hohe Spannungsverstärkung (keine hohe Frequenzen)



\subsection*{Basisschaltung}

\begin{figure}[h]
	\centering
	\includegraphics[scale=0.9]{Basisschaltung.jpg}
\end{figure}


kann bei hohe Frequenzen als Spannungs und Leistungsverstärker eingesetzt werden.


\subsection*{MOSFET/MESFET}


Der MESFET hat im Gegensatz zum MOSFET einen Metall-Halbleiter Kontakt (Schotky-Kontakt). Dadurch sind bei gleichen Abmessungen höhere Ströme als auch Frequenzen möglich.



\section{Konstanten}

\begin{align*}
e &= \unit[1,6 \cdot 10^{-19}]{V} \\
k &= \unit[1,38 \cdot 10^{-23}]{J/K} \\
\epsilon_0 &= \unit[8,854 \cdot 10^{-12}]{As / Vm} \\
R &= \unit[8,314]{J / mol \ K} \\
m_e &= \unit[9,109 \cdot 10^{-31}]{kg} \\
h &= \unit[6,626 \cdot 10^{-34}]{Js} \qquad \hbar = \frac{h}{2 \pi}
\end{align*}



















\end{document}